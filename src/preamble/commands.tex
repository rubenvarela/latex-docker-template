% =============================================================================
% commands.tex - Custom Commands and Macros
% =============================================================================
% This file contains custom LaTeX commands and macros.
% Add your own shortcuts and frequently used constructs here.
% =============================================================================

% -----------------------------------------------------------------------------
% Theorem Environments (using amsthm)
% -----------------------------------------------------------------------------
\theoremstyle{plain}                % Italic body text
\newtheorem{theorem}{Theorem}[section]
\newtheorem{lemma}[theorem]{Lemma}
\newtheorem{proposition}[theorem]{Proposition}
\newtheorem{corollary}[theorem]{Corollary}

\theoremstyle{definition}           % Normal body text
\newtheorem{definition}[theorem]{Definition}
\newtheorem{example}[theorem]{Example}
\newtheorem{exercise}[theorem]{Exercise}

\theoremstyle{remark}               % Normal body text, italic title
\newtheorem{remark}[theorem]{Remark}
\newtheorem{note}[theorem]{Note}

% Proof environment is provided by amsthm automatically

% -----------------------------------------------------------------------------
% Math Shortcuts
% -----------------------------------------------------------------------------
% Common math operators
\DeclareMathOperator{\E}{\mathbb{E}}        % Expected value
\DeclareMathOperator{\Var}{Var}             % Variance
\DeclareMathOperator{\Cov}{Cov}             % Covariance
\DeclareMathOperator{\argmax}{arg\,max}     % Argmax
\DeclareMathOperator{\argmin}{arg\,min}     % Argmin

% Number sets
\newcommand{\N}{\mathbb{N}}                 % Natural numbers
\newcommand{\Z}{\mathbb{Z}}                 % Integers
\newcommand{\Q}{\mathbb{Q}}                 % Rationals
\newcommand{\R}{\mathbb{R}}                 % Reals
\newcommand{\C}{\mathbb{C}}                 % Complex numbers

% Common math shortcuts
\newcommand{\abs}[1]{\left|#1\right|}       % Absolute value
\newcommand{\norm}[1]{\left\|#1\right\|}    % Norm
\newcommand{\inner}[2]{\langle#1,#2\rangle} % Inner product
\newcommand{\floor}[1]{\left\lfloor#1\right\rfloor}   % Floor
\newcommand{\ceil}[1]{\left\lceil#1\right\rceil}      % Ceiling

% Derivatives
\newcommand{\dd}{\mathrm{d}}                % Differential d
\newcommand{\dv}[2]{\frac{\dd #1}{\dd #2}}  % Derivative
\newcommand{\pdv}[2]{\frac{\partial #1}{\partial #2}} % Partial derivative

% Vectors and matrices
\renewcommand{\vec}[1]{\mathbf{#1}}         % Bold vectors
\newcommand{\mat}[1]{\mathbf{#1}}           % Bold matrices

% -----------------------------------------------------------------------------
% Text Shortcuts
% -----------------------------------------------------------------------------
% Commonly used text in math mode
\newcommand{\st}{\text{ s.t. }}             % "such that"
\newcommand{\ie}{\textit{i.e.}}             % i.e.
\newcommand{\eg}{\textit{e.g.}}             % e.g.
\newcommand{\etc}{\textit{etc.}}            % etc.
\newcommand{\etal}{\textit{et al.}}         % et al.

% -----------------------------------------------------------------------------
% Code and Technical Terms
% -----------------------------------------------------------------------------
% Inline code (for short snippets)
\newcommand{\code}[1]{\texttt{#1}}

% File paths
\newcommand{\filepath}[1]{\texttt{#1}}

% Keyboard keys
\newcommand{\key}[1]{\fbox{\small\texttt{#1}}}

% -----------------------------------------------------------------------------
% Figure and Table Shortcuts
% -----------------------------------------------------------------------------
% Quick figure insertion
% Usage: \quickfig{filename}{caption}{label}
\newcommand{\quickfig}[3]{%
    \begin{figure}[htbp]
        \centering
        \includegraphics[width=0.8\textwidth]{#1}
        \caption{#2}
        \label{fig:#3}
    \end{figure}
}

% -----------------------------------------------------------------------------
% TODO and Notes
% -----------------------------------------------------------------------------
% Note in margin (useful during drafting)
\newcommand{\todo}[1]{\marginpar{\footnotesize\textcolor{red}{TODO: #1}}}
\newcommand{\draftnote}[1]{\marginpar{\footnotesize\textcolor{blue}{Note: #1}}}

% Inline todo (visible in text)
\newcommand{\inlinetodo}[1]{\textcolor{red}{\textbf{[TODO: #1]}}}

% =============================================================================
% End of commands.tex
% =============================================================================
