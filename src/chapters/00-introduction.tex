% =============================================================================
% 00-introduction.tex - Introduction Chapter
% =============================================================================

\section{Introduction}
\label{sec:introduction}

Welcome to the LaTeX Template Repository. This template provides a structured approach to writing LaTeX documents with modern tooling and best practices.

\subsection{Purpose}

This template is designed to:
\begin{itemize}
    \item Provide a clean, modular document structure
    \item Include commonly used LaTeX packages pre-configured
    \item Offer build automation via Make and Python scripts
    \item Support continuous integration through GitHub Actions
\end{itemize}

\subsection{Getting Started}

To get started with this template:

\begin{enumerate}
    \item Clone or use this repository as a template
    \item Run \code{make setup} to verify your environment
    \item Edit files in the \filepath{src/} directory
    \item Run \code{make build} to compile your document
\end{enumerate}

\subsection{Document Structure}

The document is organized into several directories:

\begin{description}
    \item[\filepath{src/}] Main LaTeX source files
    \item[\filepath{src/preamble/}] Package imports, settings, and custom commands
    \item[\filepath{src/chapters/}] Individual chapter files
    \item[\filepath{src/bibliography/}] Bibliography files (\code{.bib})
    \item[\filepath{assets/}] Images, figures, and other assets
    \item[\filepath{styles/}] Custom style files (\code{.sty})
\end{description}

\subsection{Sample Content}

This template demonstrates various LaTeX features. See the test document in \filepath{tests/test\_document.tex} for comprehensive examples of:

\begin{itemize}
    \item Mathematical equations and theorem environments
    \item TikZ graphics and PGFPlots charts
    \item Professional tables with \code{booktabs}
    \item Code listings with syntax highlighting
    \item Citations and cross-references
\end{itemize}

% =============================================================================
% End of 00-introduction.tex
% =============================================================================
